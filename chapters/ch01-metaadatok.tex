
\chapter{Metaadatok}

Ebben a fejezetben a \texttt{szte-thesis} osztályban használt metaadat-parancsokat mutatjuk be.  
Ezeket tipikusan a \texttt{main.tex} elején adjuk meg, még a \verb|\begin{document}| előtt.  

\section{Alap adatok: cím, szerző, Neptun-kód}

A dolgozat legfontosabb adatai:

\begin{itemize}
  \item \verb|\title{...}| -- a dolgozat címe,
  \item \verb|\author{...}| -- a hallgató neve,
  \item \verb|\neptun{...}| -- a Neptun-kód (ha használni szeretnénk).
\end{itemize}

\noindent
A \texttt{code/ch01-metaadatok-main.tex} fájl egy minimális példa ezekre.  
A kódot a következő módon importáljuk, kikapcsolható kódként:

\inputcode[Minden kötelező metaadat megadása]{TeX}{code/ch01-metaadatok-main.tex}

\section{Fokozat és titkosság}

A dolgozat típusát a \verb|\setdegree| paranccsal adjuk meg:

\begin{itemize}
  \item \verb|\setdegree{bsc}| -- BSc szakdolgozat,
  \item \verb|\setdegree{msc}| -- MSc diplomamunka,
  \item \verb|\setdegree{tdk}| -- TDK dolgozat.
\end{itemize}

\noindent
A titkosságot a \verb|\setconfidential{true}| vagy \verb|\setconfidential{false}| parancs állítja be.  
Ha a szakdolgozat titkosított, akkor az érték \verb|true|, ellenkező esetben \verb|false|.  
A hozzá tartozó példa a \texttt{code/ch01-degree-confidential.tex} fájlban található:

\inputcode[Fokozat és titkosság beállítása]{TeX}{code/ch01-degree-confidential.tex}

\section{Szak, szakirány, tanszék}

A szakhoz és a szakirányhoz tartozó metaadatokat az alábbi parancsokkal adjuk meg:

\begin{itemize}
  \item \verb|\program{...}| -- a szak neve,
  \item \verb|\specialization{...}| -- szakirány,
  \item \verb|\department{...}| -- az első témavezető tanszéke.
\end{itemize}

\noindent
Példa ezek használatára:

\inputcode[Szak, szakirány és tanszék megadása]{TeX}{code/ch01-program-department.tex}

\section{Témavezető és egyéb szereplők}

Az első témavezető, a második témavezető és a konzulens adatai a következő parancsokkal állíthatók be:

\begin{itemize}
  \item \verb|\supervisor{...}|, \verb|\supervisortitle{...}|,\\
   \verb|\supervisordepartment{...}|,
  \item \verb|\secondsupervisor{...}|, \verb|\secondsupervisortitle{...}|, \\
  \verb|\secondsupervisordepartment{...}|,
  \item \verb|\consultant{...}|, \verb|\consultantposition{...}|,\\
   \verb|\consultantworkplace{...}|.
\end{itemize}

\noindent
A \texttt{code/ch01-supervisors.tex} fájl egy teljes példát tartalmaz ezekre:

\inputcode[Témavezetők és konzulens megadása]{TeX}{code/ch01-supervisors.tex}
