
\chapter{Irodalomjegyzék és hivatkozások kezelése}

Ebben a fejezetben azt mutatjuk be, hogyan érdemes irodalomjegyzéket készíteni \texttt{biblatex} segítségével.  
Bemutatjuk, milyen hasznos könyvek és online források érhetők el.  
Megnézzük, hogyan néz ki egy tétel a \texttt{.bib} fájlban.  
Áttekintjük, milyen lépésekkel kell fordítani a dokumentumot.  

\section{Hasznos könyvek és online források}

A \LaTeX{} „alapműve” Leslie Lamport könyve, \emph{LaTeX: A Document Preparation System}~\cite{lamport1994}.  
Ez a könyv részletesen bemutatja a rendszer működését és a tipikus dokumentumok felépítését.  

Haladóbb használathoz hasznos \emph{The \LaTeX\ Companion} második kiadása~\cite{mittelbach2004}.  
Ez a könyv rengeteg gyakorlati példát ad többek között az irodalomjegyzék-kezelésre is.  

A \texttt{biblatex} csomagnak saját, részletes kézikönyve van~\cite{biblatexmanual}.  
Ez a dokumentáció elérhető a CTAN-on.  
Ez írja le a csomag fontos opcióit és a rendelkezésre álló stílusokat.  

Online bevezetőt ad például az Overleaf „Bibliography management with biblatex” című cikke~\cite{overleaf-biblatex}.  

\section{Mi van egy \texttt{.bib} tételben?}

Az irodalomjegyzék forrásai egy \texttt{.bib} fájlban tárolódnak.  
Minden forrás egy \emph{tétel} (entry), amelynek van típusa és mezői.  
A típus lehet például \texttt{@book}, \texttt{@article} vagy \texttt{@online}.  
A mezők jellemzően a szerző, cím, év és kiadó adatokat tartalmazzák.  

Nézzük meg példaként a Lamport-könyv bejegyzését~\cite{lamport1994}.  
Egy lehetséges \texttt{@book} tétel így néz ki a \texttt{references.bib} fájlban:  

\begin{codeblock}[caption={Egyszerű \texttt{@book} tétel a \texttt{.bib} fájlban}]{TeX}
@book{lamport1994,
  author    = {Leslie Lamport},
  title     = {LaTeX: A Document Preparation System},
  edition   = {2},
  publisher = {Addison-Wesley},
  year      = {1994},
  address   = {Reading, Massachusetts},
  isbn      = {978-0201529838},
}
\end{codeblock}

A mezők szerepe az alábbi.  

\begin{itemize}
  \item \verb|lamport1994| a tétel kulcsa. Ezt használjuk a szövegben a \verb|\cite{lamport1994}| parancsban.  
  \item Az \verb|author| mező a szerző nevét tartalmazza.  
  \item A \verb|title| mező a mű címét adja meg.  
  \item Az \verb|edition| mező a kiadás számát tartalmazza.  
  \item A \verb|publisher| mező a kiadó nevét tárolja.  
  \item A \verb|year| mező a megjelenés évét adja meg.  
  \item Az \verb|address| mező a kiadás helyét tartalmazza.  
  \item Az \verb|isbn| mező a könyv ISBN azonosítója.  
\end{itemize}

Más típusú tételek hasonló felépítésűek, csak a típus és néhány mező más.  
Például egy \texttt{@manual} vagy \texttt{@online} tétel esetén az \verb|url| és az \verb|urldate| mező is szerepelhet.  

\section{Hivatkozás a szövegben}

Ha a Lamport-tétel szerepel a \texttt{.bib} fájlban, akkor a szövegben hivatkozhatunk rá.  

\begin{itemize}
  \item \verb|\cite{lamport1994}| egyszerű hivatkozást hoz létre, például: „A \LaTeX{} rendszer részletes leírását lásd \verb|\cite{lamport1994}|.”  
  \item \verb|\cite{lamport1994,mittelbach2004}| egyszerre több forrásra hivatkozik.  
\end{itemize}

A konkrét megjelenési forma a \texttt{biblatex} stílusától függ.  
A jelen sablonban az \texttt{ieee} stílus van beállítva, ezért sorszámozott, szögletes zárójeles hivatkozásokat kapunk.  

\section{Fordítási lépések \texttt{biblatex} + \texttt{biber} használatakor}

A \texttt{biblatex} alapértelmezetten a \texttt{biber} programot használja a feldolgozáshoz.  
Ez eltér a régi \texttt{bibtex}-es munkafolyamattól.  

\begin{enumerate}
  \item A preambulumban szerepeljen az \verb|\addbibresource{references.bib}| sor.  
  \item Futtassuk a fordítást: \verb|pdflatex main|.  
  \item Ezután futtassuk a \verb|biber main| parancsot.  
  \item Végül futtassuk még kétszer a \verb|pdflatex main| parancsot.  
\end{enumerate}

Az irodalomjegyzék tényleges kiíratását a \verb|\printbibliography| parancs végzi, amely a \texttt{main.tex} végén található.  
