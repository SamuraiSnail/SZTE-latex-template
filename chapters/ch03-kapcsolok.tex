
\chapter{Kapcsolók és kikapcsolható elemek}

A \texttt{szte-thesis} osztály több logikai kapcsolót definiál, amelyekkel az ábrák, táblázatok, kódrészletek, TODO-k és matematikai ábrák egyben ki- vagy bekapcsolhatók.  
Ezek az \texttt{ifthen} csomag \verb|\setboolean| parancsát használják.  

\section{Elérhető logikai kapcsolók}

A következő logikai kapcsolók érhetők el:

\begin{itemize}
  \item \texttt{showfigures} -- ábrák (\verb|safefigure|) megjelenítése,
  \item \texttt{showtables} -- táblázatok (\verb|safetable|) megjelenítése,
  \item \texttt{showcode} -- kódrészletek (\verb|codeblock| és \verb|\inputcode|) megjelenítése,
  \item \texttt{showtodos} -- \verb|\todoi| megjegyzések megjelenítése,
  \item \texttt{showmathfigures} -- matematikai ábrák (\verb|safemathfigure|) megjelenítése.
\end{itemize}

\noindent
\newpage
A \texttt{code/ch03-switches.tex} fájl bemutatja, hogyan lehet ezeket egy helyen beállítani:

\inputcode[Kapcsolók beállítása]{TeX}{code/ch03-switches.tex}

\section{TODO megjegyzések kikapcsolása}

A \verb|\todoi| parancs egy jól látható TODO dobozt jelenít meg a szövegben.  
Ha a \texttt{showtodos} logikai kapcsoló értéke \verb|false|, akkor ezek egyáltalán nem jelennek meg a végleges PDF-ben.  

\inputcode[TODO megjegyzés használata]{TeX}{code/ch03-todo-example.tex}
