
\chapter{Fedlap és nyilatkozat}

A \texttt{szte-thesis} osztály két speciális parancsot ad a fedlap és a nyilatkozat automatikus létrehozására.  
Az első a \verb|\makecover|, amely a metaadatok alapján elkészíti a dolgozat fedlapját.  
A második a \verb|\printdeclaration|, amely létrehozza a „Nyilatkozat” fejezetet.  

\section{Fedlap: \texttt{\textbackslash makecover}}

A fedlap a metaadatokból épül fel (\verb|\title|, \verb|\author|, \verb|\program|, \verb|\degreeTitle| stb.).  
A felhasználó számára mindössze annyi a teendő, hogy a metaadatok beállítása után meghívja a \verb|\makecover| parancsot a dokumentum elején.  

\begin{codeblock}[caption={Fedlap létrehozása a fő fájlban}]{TeX}
\makecover
\end{codeblock}

\section{Nyilatkozat: \texttt{\textbackslash printdeclaration}}

A \verb|\printdeclaration| parancs létrehozza a „Nyilatkozat” fejezetet.  
A szövegben egyértelműen szerepel, hogy ha a szakdolgozat titkosított, akkor a \verb|confidential| logikai kapcsoló értéke \verb|true|, egyébként \verb|false|.  

\begin{codeblock}[caption={Nyilatkozat beillesztése}]{TeX}
\printdeclaration
\end{codeblock}
